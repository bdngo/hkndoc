\documentclass[11pt, article, oneside]{memoir}
\usepackage{hkn}
\usepackage[margin=1in]{geometry}
\setlength{\parindent}{0in}


\begin{document}
	\maketitle
	
	This document serves to provide all information needed for candidates to understand everything about HKN Mu Chapter at UC Berkeley. Candidates should find this extremely helpful for navigating their candidate semester, learning about what HKN provides to the community, and finding answers to many of their questions.
    \bigbreak

    This document was originally written by Omar Yu (Fall '23 External Vice President). A special thank you to Simon Kuang and Gina Wu for the HKN Latex template.
	
	\tableofcontents*
	\newpage

    %------------------------------------- INTRODUCTION 
    
	\chapter{What is HKN?}
    \section{Introduction}
        Eta Kappa Nu (HKN) is the national Electrical Engineering and Computer Science Honor Society. The organization was founded at the University of Illinois at Urbana-Champaign (UIUC) in 1904. There are over 260 chapters worldwide, with over 200,000 members inducted. Chapter activites are organized around the recognition of academic accomplishment, the promotion of volunteer service, and the development of leadership and collaborative skills. 
        
        \bigbreak
        At UC Berkeley, we are the Mu ($\mu$) Chapter, which was founded in 1915. We are one of the most, if not the most, active chapters of HKN in the entire world, having won the Outstanding Chapter Award for the past 23 years in a row. This semester (Fall '23), we have 70+ active officers working across 10 committees. We provide services to the general EECS community, the EECS Department, the College of Engineering, and the  community at Berkeley as a whole.
        
    \section{Membership and Operations}
        Membership in HKN is extended to a certain subset of students in EECS every semester. Those who accept the invitation are known as \emph{Candidates}. They then engage in the \emph{Candidate Semester}, which gives them a full tour of what it's like to be in HKN. After completing requirements, they then initiate at the end of the semester as nationally-inducted members.

        \bigbreak

        Members can then run during Elections to become Officers or Assistant Officers, roles with required weekly/semesterly responsibilities for a specific committee and HKN as a whole. Some general services that HKN provides are:
            \begin{itemize}
                \item Exam Review Sessions for all lower-division EECS courses
                \item Free Drop-in Tutoring Hours
                \item Resume Critiques
                \item Career Fairs
                \item Department Tours
                \item EECS Day for prospective students
                \item Info Sessions with companies
                \item Post-Exam Fruit Snack Attacks
                \item Going Down the EECS Stack Decal
                \item De-Stress Week
                \item EECS Community Week 
                \item And much more!
            \end{itemize}
        
        The listed services are just a peek into what HKN does as an organization. Members also have participate in semesterly retreats, socials, food hangouts, and other events! It's definitely a work hard, play hard environment. 

    % \section{Relevant Links}
    %     Here's a list of relevant links you might make use of:
    %     \begin{itemize}
    %         \item Website: \href{https://hkn.mu}{hkn.mu}
    %         \item New Website for Candidates: \href{https://hkn.mu/dev}{hkn.mu/dev}
    %         \item Event Calendar (must be logged in to see events): \href{https://hkn.mu/cal}{hkn.mu/cal}
    %         \item Candidate Portal: \href{https://hkn.mu/candportal}{hkn.mu/candportal}
    %         \item TBD?
    %     \end{itemize}

    %------------------------------------- CANDIDATE SEMESTER 
    \newpage
    \chapter{Candidate Semester}

    \section{Being a Candidate}
        Congratulations on being invited to HKN! Candidacy is extended to the top third of students with senior standing and the top fourth of students with junior standing, with standing being determined by units completed. 
        
        \bigbreak

        Being a candidate means that you'll go through the \textbf{\emph{Candidate Semester}}, which is the semester that you are invited to HKN. The Candidate Semester is designed for you to learn about HKN and have a bunch of fun meeting people. You'll get to attend many events ranging from fun to community service, be a part of a big-little system, have the opportunity to work with a committee, and so much more. This section is dedicated to providing you with all the information you'll need to make the most out of your Candidate Semester!

        \bigbreak

        \textbf{Your primary point of contact during the Candidate Semester will be the External Vice President (EVP - Omar Yu Fall '23)}, but you will have a couple of other assigned people to talk to and ask questions if you ever need to.


    \section{Requirements}
        As a candidate, you'll have some requirements to complete in order to initiate as a member at the end of the semester. While this might seem like a LOT of work at first glance, we promise that it's not and is actually super fun! Aside from  General Meetings, these requirements should really only take about an hour of your time per week of the Candidate Semester.
        
        \bigbreak

        \textbf{Now, let's actually get into the requirements:}
            \begin{itemize}
                \item \textbf{Attend Candidate Meeting (CM)}: You've probably already done this if you're reading this. \textbf{If you couldn't make CM, reach out to EVP for alternatives if you haven't already.}
                \item \textbf{Attend all 3 General Meetings (GMs)}: These are integral to your Candidate Semester experience and are the most important events to make. \textbf{If you can't make one of these, let your EVP know immediately.}
                \item \textbf{Pay 2 sets of dues}: There will be 2 sets of dues throughout the Candidate Semester. The first set is mandated by Nationals and is the fee to initiate your life-long membership in HKN. This money goes straight to Nationals and we at Berkeley personally do not gain anything from or take any portion of this fee. The second set is to help pay for Banquet at the end of the semester, where we celebrate the new initiates, give out awards, and more. \textbf{If you have any questions, comments, or concerns regarding these dues, please reach out to your EVP and we'll work something out.}
                \item \textbf{Attend 3 Fun events}: Fun events are exactly what they sound like - fun! There will be an multiple, varied activities every week for you to take part in and enjoy.
                \item \textbf{Attend 1 Big Fun event}: Big Funs are Funs but bigger. These events span from going to SF, having a puzzle-solving race across campus, participating in large-scale field events, and more.
                \item \textbf{Attend 1 Community event}: Community events are meant for you to hang out with the greater EECS community or the general community at Berkeley. These range from chill make-your-own boba sessions, playing board games, collaborating with other organizations, and more.
                \item \textbf{Attend 1 Service event}: Service is a core pillar of HKN. You'll have the opportunity to help out at our annual EECS Day for prospective high school students, assist the Berkeley Project, make/hand out sandwiches for the homeless, and more.
                \item \textbf{Attend 1 Officer Hangout}: Our officers will be hosting their own hangouts for you to learn more about them and have some fun! \textbf{You must attend a hangout that is hosted by at least one officer who isn't an officer in your committee or your byte.}
                \item \textbf{Complete 1 Officer Challenge}: Officers will periodically be giving challenges out in the \#challenges channel in the Candidate Slack. We'll also have some dedicated time during General Meetings for challenges to be given out and taken. \textbf{Your challenge must be issued by an officer who is not your byte.}
                \item \textbf{Complete 3 Bit-Byte Hangouts}: As part of your Candidate Semester, you'll be part of a Bit-Byte group. Hang out with them! More details about Bit-Byte groups can be found in the Bit-Byte section below.
                \item \textbf{Complete your Candidate Project}: As part of your Candidate Semester, you'll be assigned to one of nine committees. You'll be given a Candidate Project to complete, which is all about learning and getting involved in the work your committee does. More details about the this can be found in the Candidate Project section below.
                \item \textbf{Attend Candidate Check-in}: During the later half of the Candidate Semester, you will be required to meet with the External Vice President for a short 15 minute appointment to discuss how your Candidate Semester is going, how you're liking HKN, and whether or not you're interested in running for officership. This is just a chill check-in to make sure that you're on track to initiate and answer any questions or concerns you might have.
                \item \textbf{Complete the Candidate Quiz}: There will be a short Candidate Quiz to check that you are involved with HKN and its current members. We promise it's very easy! Answers to the quiz can be found in this document and other relevant links that you'll often use throughout the Candidate Semester. If you find that you're struggling to find any of the answers, you can ask an officer/member to help you out.
            \end{itemize}


            \textbf{There's a couple of steps for having something count towards the completion of a requirement. Here's what they are:}
            \begin{enumerate}
                \item \href{https://hkn.mu/cal}{RSVP to an event on the calendar} corresponding to the type of requirement you're working on.
                \item Attend the event (lol).
                \item At the end of the event, you will either be manually checked off by an officer or there will be attendance/check-in password. This will be given out by the officer(s) hosting the event at the end of the event. Enter this password in the RSVP page of the event to get credit for it.
            \end{enumerate}
    
            To see your progress towards completing your requirements, check out the \href{https://hkn.mu/candportal}{Candidate Portal} on the HKN Dev website. \textbf{The deadline to complete all of your requirements is Root Beer Float Night (RBFN), which takes place in the last week of November.} If you ever find that you are struggling to finish some requirements and are worried that you won't be able to complete some of them, please reach out to the current External Vice President and we'll work something out.
    
            \bigbreak


    \section{Bit-Byte Groups}
        A core part of the Candidate Semester involves being in a \textbf{Bit-Byte Group}! This is HKN's big-little system, where Candidates are Bits and Members are Bytes. As a Bit, you'll be matched with 2-4 other Bits and 2 Bytes. As mentioned in the requirements, there are 3 required Bit-Byte Hangouts. These will be organized by your Bytes or you can suggest what you want your group to do as well!
        
        \bigbreak

        The goal of Bit-Byte Groups is for you to have a little family you can reach out to go to events together, ask questions about the Candidate Semester, and have members in HKN that you can always reach out to. During/after Candidate Meeting, you'll learn about the Bytes you can be a Bit under and will fill out a preference form to get matched. Then, at General Meeting 1, you'll get to meet your Bit-Byte Group for the semester! This semester, we'll be having Bit-Byte challenges for your group to complete and gain points over the semester. Some events will also be focused specifically on putting groups against other groups in fun games.
        
        \bigbreak

        If you ever encounter any issues with your Bit-Byte group, such as inactivity, feel free to reach out to EVP to get the problem resolved.

    \section{Candidate Project}
        As mentioned in the requirements, one of your biggest tasks as a Candidate is to complete your Candidate Project. The goal of this requirement is to give you a chance of contributing to the services that HKN provides to the community, the department, and its members. What your project entails depends on which committee you are assigned to! To see specific committees and the projects they are offering this semester, please see the Committees section later in this document.
        
    \section{Initiation and Banquet}
        The Candidate Semester culminates in Initation and Banquet. More details about these events will be released during General Meeting 2 and General Meeting 3.


    \section{Elections}
        Elections follows the day after Initiation and Banquet. While you will be inducted as a lifetime member of HKN during Initiation, if you really enjoyed your time as a Candidate, want to take on more responsibilities, and help contribute to a committee, you can run for officership at Elections. \textbf{This is entirely optional and will not affect your membership in HKN in any way if you do not choose to run.} Running for officership is a commitment for the next semester, i.e. if you run and get a position during Fall '23 Elections, you will be an officer during Spring '24. Every candidate, officer, and member is encouraged to come to elections to participate and say their thoughts for people running.
        
        \bigbreak

        Elections begins in the morning, ends in the early evening, and transitions into \emph{Midnight Meeting} at night. Midnight Meeting is where all the newly-elected officers come and have their first meeting to discuss overall sentiment of the semester and new initiatives/changes for the following semester.

        \bigbreak
        
        Elections is led by the President, External Vice President, and Recording Secretary. For each committee, including Executive, we do the following process:
        \begin{enumerate}
            \item We first ask for any nominations. If you want to run for officership, tell your fellow Bits and your Bytes to nominate you during the nomination phase. You can also nominate yourself if you're not comfortable asking others.
            \item For those who were nominated, we have two rounds of asking them to confirm if they truly want to run for the position. In the first round, you can either accept, defer, or reject, with defer meaning you will give your final say to accept or reject in the second round.
            \item For those who accept to run for the position, we ask them to come up to the front of the room and write their commitments for the following semester. This includes classes, other club commitments, research, work, etc. 
            \item We then ask a series of questions, ranging from why they think they are a good fit for the committee, what new initiatives they might bring, etc. Each person is given a time limit to respond to each question. Those in the audience are also allowed to ask questions, and these typically come from older officers who have been on the committee for many semesters.
            \item After enough questions are asked and we feel that we are satisfied with the responses, we ask the people running to temporarily leave the room. Those in the Elections room then deliberate on the candidates and their responses, since there are limited officer spots and there are typically more candidates than available spots.
            \item Once deliberation has concluded, the audience votes for the members they think should be an officer. Those with the highest votes become officers for however many spots there are.
            \item We then bring back the candidates into the room, congratulate them for running, and announce who won. We then repeat the process with the remaining committees.
        \end{enumerate}

        If you ever have questions about officership or Elections, feel free to ask your Byte or any other member you feel comfortable chatting with! During the later half of the Candidate Semester, you'll have a Candidate Check-in with EVP to discuss if you're interested in any committees for officership!


    %------------------------------------- COMMITTEES

    \newpage
    \chapter{Committees}

    For a list of current committee officers, please see \href{https://hkn.mu/about/officers}{hkn.mu/about/officers}

    %------------------------------------- EXECUTIVE
    \section{Executive (Exec)}
    \subsection{Role and Officer Responsibilities}
        Runs HKN, plans all meetings, oversees all club activities, interfaces with the EECS department, holds the Candidate Semester, provide member events/games, evaluates member health, corresponds with Nationals regarding chapter events, checks in with all committees, and much more. 

        \begin{itemize}
            \item \textbf{President (Pres)}: Oversees the general operation of HKN's activities and organization health. Leads the weekly general officer meetings and does whatever is necessary to help make sure other officers can get duties done. Meets with the heads of other clubs in EECS and Engineering to discuss collaborations and events. Has a good idea of the inner-workings of the organization and interfaces with other societies and the department. The president must have served at least one semester as an executive.
            \item \textbf{External Vice President (EVP)}: Has jurisdiction and responsibility for the candidate class including determining policies and requirements for initiation. Attends the vast majority of events and is the face of HKN for the incoming candidate class: this includes Fun, Big Fun, service, Professional Development events, and more. Should know the entire candidate class well by the end of the semester. Leads CM and GMs. Works very closely with the Corresponding Secretary to manage Candidate Semester logistics. Takes over presidential duties in the absence of the President.
            \item \textbf{Internal Vice President (IVP)}: Has responsibility for HKN members, which involves keeping in touch with members, maintaining committee health, and hosting member events. Keeps in contact with alumni for professional development and industrial relation opportunities. Plans gifts and photoshoots for graduating members. 
            \item \textbf{Recording Secretary (RSec)}: Records and emails out attendance and minutes of weekly officer meetings. Records member attendance at candidate meetings, including CM and GMs. Records member feedback (in the form of $+/-/\Delta$'s) throughout the semester. Makes contact cards at the beginning of the semester. Records elections proceedings, including who runs for what position and what time each election starts at, during elections. Handles appreciation notes for members and candidates. 
            \item \textbf{Corresponding Secretary (CSec)}: Assists the External Vice President with internal bookkeeping and logistics of the candidate semester. Officially initiates candidates with HKN Nationals. Handles correspondence with HKN Nationals in regards to member and officer records. Ensures that all email correspondence gets answered in a timely manner. Organizes the department course evaluations for all EECS classes. Updates course surveys for EECS classes for the HKN website.
            \item \textbf{Departmental Relations (Deprel)}: Liaises with the EECS department and College of Engineering on behalf of HKN. Organizes events which are co-sponsored by the department. Coordinates student participation in department committees. Represents HKN at department meetings. Delivers a presentation at the annual Faculty Retreat and the Undergraduate Town Hall in the Spring, and assists with Cal Day. Develops the undergraduate EECS survey with the department in the Fall. Organizes and runs tours for the department. Finds faculty speakers for the general meetings.
            \item \textbf{Treasurer (Treas)}: Prepares budget for the semester after communicating with each committee about their needs. Processes all expenses and receipts through the semester, making sure the chapter stays within budget. Collects on outstanding accounts. Works on finding new sources of funding for HKN. Picks up snail mail from 253 Cory. 
        \end{itemize}

    \subsection{Candidate Project}
        N/A.

    \bigbreak

    %------------------------------------- ACTIVITIES
    \section{Activities (Act)}
    \subsection{Role and Officer Responsibilities}
        Responsible for planning, designing, and hosting Candidate Semester Fun and Big Fun events. Works with Exec to hold semesterly retreats. Also organizes member events with Internal Vice President.

    \subsection{Candidate Project}
        Candidates in Act will be working closely with officers to give the the full event-planning experience. They'll form groups to work together on creating, organizing, and hosting their own Fun event. Along with this, they will be coming up with and administering 2 Bit-Byte group challenges. Act officers will help by overseeing logistics, giving advice, and more. 
    
    \bigbreak

    %------------------------------------- BRIDGE
    \section{Bridge}
    \subsection{Role and Officer Responsibilities}
        Records the history of HKN through writing and images. Takes pictures/videos at every event. Maintains the online HKN collection of photos on Flickr and Google Drive. Updates the photos on the office bulletin boards/walls. Creates retreat video, CM video, Banquet video, and member reacts on Slack.

    \subsection{Candidate Project}
        Candidates in Bridge have two choices for their Candidate Project. They can do one of the following (or both if they're feeling ambitious!):
        \begin{enumerate}
            \item Take fun and memorable photos/videos at 5 events throughout the Candidate Semester.
            \item Work on a bigger project or do a couple of smaller projects, including but not limited to: officer photos, making a part of Banquet video, creating Slackmojis, and updating the office photo walls.
        \end{enumerate}

    \bigbreak

    %------------------------------------- COMPUTING SERVICES

    \section{Computing Services (Compserv)}
    \subsection{Role and Officer Responsibilities}
        Builds the HKN website and tools needed by other officers. Designs and implements new projects or feature requests. Maintains the computing resources of HKN (e.g. the office desktops and servers). As HKN works to provide new opportunities to EECS students, this committee is critical to the success of many other committees' rollout of new services.

    \subsection{Candidate Project}
        Candidates in Compserv will be paired up with a Compserv officer and will help contribute to developing the new HKN website at \href{https://hkn.mu/dev}{hkn.mu/dev}! For candidates with little to no prior experience, officers will be guiding them through web development basics as well. Some things you can contribute to are:
        \begin{itemize}
            \item Fix Github issues identified by officers (or finding one on your own).
            \item Write test cases for scripting and other applications.
            \item Create UI/UX designs for new components/pages of the website.
        \end{itemize} 

        Compserv candidates will also be expected to come to at least one officer meeting and at least one work session.

    \bigbreak

    %------------------------------------- DECAL
    \section{Decal}
    \subsection{Role and Officer Responsibilities}
        Runs HKN's Decal, "Going Down the EECS Stack", which covers a wide variety of topics in EECS. Responsible for organizing and developing course material, finding instructors within or outside HKN for each topic, advertising, organizing a semesterly field trip for students, and ensuring that the class runs smoothly every week. Handles administrative duties, as well as teaching duties if needed. The Decal's instructor of record is Professor Anant Sahai.

    \subsection{Candidate Project}
        Candidates in Decal will be learning about how the Decal is ran and given a chance to contribute. This project is flexible in that you can choose what fits your skillset best. Some options are:
        \begin{itemize}
            \item Shadow officers in terms of assisting with reaching out to lecturers, attending lectures, etc.
            \item Create a new (interesting) HW/Lab assignment(s).
            \item Create/maintain master contact list of lecturers.
            \item Create a new lecture topic and slides.
        \end{itemize}

    \bigbreak

    %------------------------------------- INDUSTRIAL RELATIONS
    \section{Industrial Relations (Indrel)}
    \subsection{Role and Officer Responsibilities}
        Provides industry opportunities to the EECS student body. Acts as the industry-facing branch of HKN. Organizes HKN's career fairs, info-sessions, company tours and dinners, and other industry-related events. Collects and maintains the HKN resume book.

    \subsection{Candidate Project}
        Candidates in Indrel will be working closely with officers to learn how HKN interacts with companies and hosts industry events. This project is a point-based system where candidates can choose from the following options to hit a certain threshold to complete their project:
        \begin{itemize}
            \item Assist an Indrel officer with the planning of a company info-session.
            \item Design a possible new company event for next semester and future years.
            \item Work with the Student Relations committee to market the Global Startup Fair.
            \item Find recruiter contacts for a certain amount of companies.
            \item Create new fundraising initiatives.
            \item Attend Indrel officer meeting(s).
        \end{itemize}
    \bigbreak

    %------------------------------------- PROFESSIONAL DEVELOPMENT
    \section{Professional Development (Prodev)}
    \subsection{Role and Officer Responsibilities}
        Runs professional development activities for HKN members and the EECS community. Organizes educational presentations and panels. Holds professional development office hours for resume reviews and interview practice for members and general students alike. Facilitates HKN alumni-member industry referral program and builds research pipeline.

    \subsection{Candidate Project}
        Candidates in Prodev will be working closely with officers to help provide professional development events. This project is also a point-based system where candidates can choose from the following options to hit a certain threshold to complete their project:
        \begin{itemize}
            \item Shadow resume reviews done by officers.
            \item Help organize events such as interview and resume workshops.
            \item Develop writeups for company recruiting processes, job pathways, etc.
            \item Create LeetCode walkthroughs.
            \item Participate in officer meetings.
        \end{itemize}

    \bigbreak

    %------------------------------------- SERVICE
    \section{Service (Serv)}
    \subsection{Role and Officer Responsibilities}
        Responsible for setting up and running the community service projects for the semester. This involves designing and planning the events, doing outreach, developing content, and running the events on the day of. Events include the Berkeley Project, EECS Day, and more. Current outreach audiences include middle/highschool students, local Berkeley groups, and senior citizens.

    \subsection{Candidate Project}
        Candidates in Serv will be assisting officers in planning, organizing, hosting, and helping out at service events. Events that the Serv committee plans to engage in this semester include:
        \begin{itemize}
            \item (Jr) EECS Day EE/CS labs
            \item EECS Day tour
            \item EECS Day panel 
            \item Berkeley City College EECS Workshop 
            \item Food kitchen event planning 
            \item The Berkeley Project 
            \item Trash cleanup/weed pulling
            \item Sandwiches and cards for the homeless
            \item Any ideas that candidates have!
        \end{itemize}

    \bigbreak

    %------------------------------------- STUDENT RELATIONS
    \section{Student Relations (Studrel)}
    \subsection{Role and Officer Responsibilities}
        Provides services and hosts events for the EECS student community at large, including recurring events like the EECS Course Planning Workshop, destress events such as Board Game Night, and creates new events as they see fit. Collaborates with other organizations as needed for events. Handles the publicity needs of HKN, including the HKN Instagram, which ensures that students know about our services and events. Handles merchandise, such as EECS apparel and fun items. Holds fruit snack attack runs for undergraduate EECS courses.

    \subsection{Candidate Project}
        Candidates in Studrel will be working closely with officers to help provide community events. This project is also a point-based system where candidates can choose their tasks to hit a certain threshold to complete their project. Candidates must complete at least one task from each of the two categories, with the following options.

        \bigbreak

        Graphic Design Category:
        \begin{itemize}
            \item Design Committee Spotlight graphics to be posted on the HKN Instagram.
            \item Design Instagram post for events occurring throughout the semester.
            \item Design HKN Merch with committee approval.
        \end{itemize}

        Event Planning Category:
        \begin{itemize}
            \item Help organize a Community event for the general EECS community.
            \item Help distribute fruit snacks after lower-division exams.
            \item Attend a Studrel officer meeting.
        \end{itemize}

    \bigbreak

    %------------------------------------- TUTORING
    \section{Tutoring}
    \subsection{Role and Officer Responsibilities}
    Investigates and experiments with any academic services and ideas that can help the EECS student body. Takes charge of scheduling and quality controlling office tutoring hours. Organizes review sessions for lower-division/select upper-division courses and finds members to present them. Maintains and updates the course guide and the exam archive. Oversees the upkeep of both Soda and Cory offices, ensuring the offices are kept clean and in working order. Ensures that inventory from committees is properly maintained.

    \subsection{Candidate Project}
        Candidates in Tutoring will be working closely with officers to help provide academic services. This project is also a point-based system where candidates can choose from the following options to hit a certain threshold to complete their project:
        \begin{itemize}
            \item Hold a Tutoring Hour.
            \item Help revamp the office with decorations.
            \item Update course guides with relevant and recent information.
            \item Revamp review session presentation slides.
            \item Develop academic study resources, i.e. crib sheets, cheat sheets, etc.
            \item Organize a hangout for Tutoring officers and candidates.
            \item Attend Tutoring officer meeting.
            \item Build your own project! 
        \end{itemize}
    
    \bigbreak
    
    %------------------------------------- TERMS/ACRONYMS


    \newpage
    \chapter{Common Terms, Acronyms, and Abbreviations}
    \begin{itemize}
        \item \textbf{HKN}: Eta Kappa Nu
        \item \textbf{HM}: HKN Meeting - weekly meeting for members to provide committee updates, talk about big events, and play games.
        \item \textbf{DM}: Discussion Meeting - bi-weekly-ish meeting for members to discuss new initiatives, issues, and other topics surrounding both the club and the EECS department.
        \item \textbf{EM}: Executive Meeting - weekly meeting for Exec to discuss important administrative/logistical topics, general club activities, and check in on committees/general member health.
        \item \textbf{CM}: Candidate Meeting
        \item \textbf{GM}: General Meeting
        \item \textbf{RBFN}: Root Beer Float Night/Really Big Fun Night - one of the last events of the Candidate Semester that occurs in the final week of it. Typically marked as the deadline for candidate requirements to be completed by.
        \item \textbf{Pres}: President
        \item \textbf{EVP}: External Vice President
        \item \textbf{IVP}: Internal Vice President
        \item \textbf{RSec}: Recording Secretary
        \item \textbf{CSec}: Corresponding Secretary
        \item \textbf{Deprel}: Department Relations
        \item \textbf{Treas}: Treasurer
        \item \textbf{Act}: Activities Committee
        \item \textbf{Compserv}: Computing Services Committee
        \item \textbf{Indrel}: Industrial Relations Committee
        \item \textbf{Prodev}: Professional Development Committee
        \item \textbf{Serv}: Service Committee
        \item \textbf{Studrel}: Student Relations Committee
    \end{itemize}
    
\end{document}