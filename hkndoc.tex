\documentclass[11pt, article, oneside]{memoir}
\usepackage{hkn}
\usepackage[margin=1in]{geometry}
\setlength{\parindent}{0in}


\begin{document}
	\maketitle
	
	This document serves to provide all information needed for candidates to understand everything about HKN Mu Chapter at UC Berkeley. Candidates should find this extremely helpful for navigating their candidate semester, learning about what HKN provides to the community, and finding answers to many of their questions.
    \bigbreak

    This document was originally written by Omar Yu (Fall '23 External Vice President). A special thank you to Simon Kuang and Gina Wu for the HKN Latex template.
	
	\tableofcontents*
	\newpage

    %------------------------------------- INTRODUCTION 
    
	\chapter{What is HKN?}
    \section{Introduction}
        Eta Kappa Nu (HKN) is the national Electrical Engineering and Computer Science Honor Society. The organization was founded at the University of Illinois at Urbana-Champaign (UIUC) in 1904. There are over 260 chapters worldwide, with over 200,000 members inducted. Chapter activites are organized around the recognition of academic accomplishment, the promotion of volunteer service, and the development of leadership and collaborative skills. The emblem/symbol of HKN is the wheatstone bridge and its primary colors are navy blue and scarlet. \emph{(Candidates should find this information helpful for something!)}
        
        \bigbreak
        At UC Berkeley, we are the Mu ($\mu$) Chapter, which was founded in 1915. We are one of the most, if not the most, active chapters of HKN in the entire world, having won the Outstanding Chapter Award for the past 23 years in a row. This semester (Fall '23), we have 70+ active officers working across 10 committees. We provide services to the general EECS community, the EECS Department, the College of Engineering, and the  community at Berkeley as a whole.

    \section{History}
        
    \section{Current Operations}
    \section{Officership}
    \section{Relevant Links}

    %------------------------------------- CANDIDATE SEMESTER 
    \newpage
    \chapter{Candidate Semester}

    \section{Being a Candidate}
        Congratulations on being invited to HKN! Candidacy is extended to the top third of students with senior standing and the top fourth of students with junior standing, with standing being determined by units completed. 
        
        \bigbreak

        Being a candidate means that you'll go through the \textbf{\emph{Candidate Semester}}, which is the semester that you are invited to HKN. The Candidate Semester is designed for you to learn about HKN and have a bunch of fun meeting people. You'll get to attend many events ranging from fun to community service, be a part of a big-little system, have the opportunity to work with a committee, and so much more. This section is dedicated to providing you with all the information you'll need to make the most out of your Candidate Semester!


    \section{Requirements}
        As a candidate, you'll have some requirements to complete in order to initiate as a member at the end of the semester. While this might seem like a LOT of work at first glance, we promise that it's not and is actually super fun! Aside from  General Meetings, these requirements should really only take about an hour of your time per week of the Candidate Semester.
        
        \bigbreak

        There's a couple of steps for having something count towards the completion of a requirement. Here's what they are:
        \begin{enumerate}
            \item \href{https://dev-hkn.eecs.berkeley.edu/events/}{RSVP to an event on the calendar} corresponding to the type of requirement you're working on.
            \item Attend the event (lol).
            \item At the end of the event, you will either be manually checked off by an officer or there will be attendance/check-in password. This will be given out by the officer(s) hosting the event at the end of the event. Enter this password in the RSVP page of the event to get credit for it.
        \end{enumerate}

        To see your progress towards completing your requirements, check out the \href{https://dev-hkn.eecs.berkeley.edu/cand/}{Candidate Portal} on the HKN Dev website. \textbf{The deadline to complete all of your requirements is Root Beer Float Night (RBFN), which takes place in the last week of November.} If you ever find that you are struggling to finish some requirements and are worried that you won't be able to complete some of them, please reach out to the current External Vice President (Omar Yu for Fall '23) and we'll work something out.

        \bigbreak

        Now, let's actually get into the requirements:
            \begin{itemize}
                \item \textbf{Attend Candidate Meeting (CM)}: You've probably already done this if you're reading this.
                \item \textbf{Attend all 3 General Meetings (GMs)}: These are integral to your candidate semester experience and are the most important events to make. \textbf{If you can't make one of these, let your EVP know immediately.}
                \item \textbf{Pay 2 sets of dues}: There will be 2 sets of dues throughout the Candidate semester. The first set is mandated by Nationals and is the fee to initiate your life-long membership in HKN. This money goes straight to Nationals and Mu Chapter (us at Berkeley) personally does not gain anything from or take any portion of this fee. The second set is to help pay for Banquet at the end of the semester, where we celebrate the new initiates, give out awards, and more. \textbf{If you have any questions, comments, or concerns regarding these dues, please reach out to your EVP and we'll work something out.}
                \item \textbf{Attend 3 Fun events}: Fun events are exactly what they sound like - fun! There will be an almost seemingly-infinite amount of varied activities for you to take part in and enjoy.
                \item \textbf{Attend 1 Big Fun event}: Big Funs are Funs but bigger. These events span from going to SF, having a puzzle-solving race across campus, participating in large-scale field events, and more.
                \item \textbf{Attend 1 Community event}: Community events are meant for you to hang out with the greater EECS community or the general community at Berkeley. These range from chill make-your-own boba sessions, playing board games, collaborating with other organizations, and more.
                \item \textbf{Attend 1 Service event}: Service is a core pillar of HKN. You'll have the opportunity to help out at our annual EECS Day for prospective high school students, assist the Berkeley Project, make/hand out sandwiches for the homeless, and more.
                \item \textbf{Attend 1 Officer Hangout}: Our officers will be hosting their own hangouts for you to learn more about them and have some fun! \textbf{You must attend a hangout that is hosted by at least one officer who isn't an officer in your committee or your byte.}
                \item \textbf{Complete 1 Officer Challenge}: Officers will periodically be giving challenges out in the \#challenges channel in the candidate Slack. We'll also have some dedicated time during General Meetings for challenges to be given out and taken. \textbf{Your challenge must be issued by an officer who is not your byte.}
                \item \textbf{Complete 3 Bit-Byte Hangouts}: As part of your Candidate Semester, you'll be part of a Bit-Byte group. Hang out with them! More details about Bit-Byte groups can be found in the Bit-Byte section below.
                \item \textbf{Complete your Candidate Project}: As part of your Candidate Semester, you'll be assigned to one of our nine committees. You'll be given a Candidate project to complete, which is all about learning and getting involved in the work your committee does. More details about the this can be found in the Candidate Project section below.
                \item \textbf{Complete the Candidate Quiz}: There will be a short Candidate Quiz to check that you are involved with HKN and its current members. We promise it's very easy! Answers to the quiz can be found in this document and other relevant links that you'll often use throughout the candidate semester. If you find that you're struggling to find any of the answers, you can ask an officer/member to help you out.
            \end{itemize}



    \section{Bit-Byte Groups}
    \section{Candidate Project}
    \section{Initiation and Banquet}
    \section{Elections}


    %------------------------------------- COMMITTEES

    \newpage
    \chapter{Committees}
    \section{Executive (Exec)}
    \subsection{Role}

    \subsection{Officer Responsibilites}

    \subsection{Candidate Project}

    \bigbreak



    \section{Activities (Act)}
    \subsection{Role}

    \subsection{Officer Responsibilites}

    \subsection{Candidate Project}

    \section{Bridge}
    \subsection{Role}

    \subsection{Officer Responsibilites}

    \subsection{Candidate Project}

    \bigbreak


    \section{Computing Services (Compserv)}
    \subsection{Role}

    \subsection{Officer Responsibilites}

    \subsection{Candidate Project}

    \bigbreak

    
    \section{Decal}
    \subsection{Role}

    \subsection{Officer Responsibilites}

    \subsection{Candidate Project}

    \bigbreak

    
    \section{Industrial Relations (Indrel)}
    \subsection{Role}

    \subsection{Officer Responsibilites}

    \subsection{Candidate Project}

    \bigbreak

    
    \section{Professional Development (Prodev)}
    \subsection{Role}

    \subsection{Officer Responsibilites}

    \subsection{Candidate Project}


    \bigbreak

    
    \section{Service (Serv)}
    \subsection{Role}

    \subsection{Officer Responsibilites}

    \subsection{Candidate Project}


    \bigbreak

    
    \section{Student Relations (Studrel)}
    \subsection{Role}

    \subsection{Officer Responsibilites}

    \subsection{Candidate Project}


    \bigbreak

    
    \section{Tutoring}
    \subsection{Role}

    \subsection{Officer Responsibilites}

    \subsection{Candidate Project}

    
    \bigbreak
    
    %------------------------------------- TERMS/ACRONYMS


    \newpage
    \chapter{Common Terms, Acronyms, and Abbreviations}
    \begin{itemize}
        \item \textbf{HKN}: Eta Kappa Nu
        \item \textbf{HM}: HKN Meeting - weekly meeting for members to provide committee updates, talk about big events, and play games.
        \item \textbf{DM}: Discussion Meeting - bi-weekly-ish meeting for members to discuss new initiatives, issues, and other topics surrounding both the club and the EECS department.
        \item \textbf{EM}: Executive Meeting - weekly meeting for Exec to discuss important administrative/logistical topics, general club activities, and check in on committees/general member health.
        \item \textbf{CM}: Candidate Meeting
        \item \textbf{GM}: General Meeting
        \item \textbf{RBFN}: Root Beer Float Night/Really Big Fun Night - one of the last events of the Candidate Semester that occurs in the final week of it. Typically marked as the deadline for candidate requirements to be completed by.
        \item \textbf{EVP}: External Vice President
        \item \textbf{IVP}: Internal Vice President
        \item \textbf{RSec}: Recording Secretary
        \item \textbf{CSec}: Corresponding Secretary
        \item \textbf{Deprel}: Department Relations
        \item \textbf{Act}: Activities Committee
        \item \textbf{Compserv}: Computing Services Committee
        \item \textbf{Indrel}: Industrial Relations Committee
        \item \textbf{Prodev}: Professional Development Committee
        \item \textbf{Serv}: Service Committee
        \item \textbf{Studrel}: Student Relations Committee
    \end{itemize}
    
\end{document}